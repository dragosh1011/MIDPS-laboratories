\section*{Concluzie}
\phantomsection

Elaborarea acestei lucrări de laborator mi-a adus o experiență nouă în domeniul programării. Această lucrarea de laborator reprezintă prima mea aplicație mobilă și mi-a adus multe cunoștințe noi. Deși tehnologiile utilizate nu include un limbaj nativ pentru una din platforme, acestea au avantajul că pot fi distribuite pe mai multe din ele, incluzînd ios, adroid și windows, probabil e compatibil și cu alte platforme mai puțin utilizate la moment. 

Alegerea de a crea o aplicație cu implementarea tehnicii promodoro a adus după sine necesitate de informare ce reprezintă această tehnică și de analiză cum ar putea fi ea implementată. Lipsa cunoștințelor în angular 2 și tipescript mi-au îngreunat procesul de dezvoltare la începutul aplicației și deși a fost un moment în care mă gândeam să trec aplicația pe ionic(1) care folosește respectiv prima versiune de angular, am dus aplicația la bun sfîrșit cu tehnologiile inițiale alese. 

A fost o experiență bună de a acumula exeriență de utilizare a unor frameworkuri necunoscute și de îmbunătățire a cunoștințelor în limbajele indicate mai sus. 
Am întîlnit unele problme la emularea aplicațiilor atît pe ios cît și pe android. La ios am avut o problemă cu keybordul care nu apărea pe ecranul dispozitivului, iar pe android am piedut destul de mult timp pentru a instala sdk și alte pachete necesare. Am reușit să trec peste acestea căutând soluțiile pe diferite forumuri și repositoriile github relaționate de ionic 2.

Pe parcursul elaborării mi-a apărut întrebarea de ce producătoii nu optează pentru o asemenea tehnologie care permite utilizarea aplicației pe diferite platforme, însă după problemele întîlnite și o disuție cu colegii care au experiență în domeniu am înțeles că stabilitatea unei asemenea aplicații are destul de multe semne de întrebare, de aceea cred că în viitor voi opta pentru a face o aplicație pentru o sinură platformă și utilizînd tehnologiile reomandate pentru aceasta.

\clearpage
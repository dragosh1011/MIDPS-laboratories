\section*{Concluzie}
\phantomsection

Dezvoltarea undei aplicatii web este intotdeauna o sarcina interesanta pentru mine, desi partea de html si mai cu seama, stilurile sigur nu sunt punctul meu forte si nici nu ma bucur prea mult cand sunt nevoit sa interactionez cu aceste parti. In schimb javascriptul imi place si faptul ca se dezvolta foarte mult si cunoaste o crestere mare in ultimi ani nu poate decat sa ma bucure. 
WebStorm este un ide destuld e raspindid intre front-end developerii si nu numai. Este comod si ofera atit configurari pre stabilitate care sa te ajute, cat si posibilitatea de a configura multe tooluri de care poti avea nevoi in dezvoltarea web. Crearea unei aplicatii angular dupa o perioada indelungata in care nu am interactionat a fost destul de interesant si ma bucur ca cunostintele acumulate anterior mi-au fost de ajutor. 
Realizarea unui api este mereu o provocare interesanta, indiferent de dimensiunile sale. Multimea de tehnologii si pachete din care poti alege la realizarea unei aplicatii  node te pune de multe oriin dificultate, pentru ca daca vrei sa eveti problemele viitoare, trebuie sa faci o alegere buna de la inceput. 
Prin utilizara undei baze de date in mongodb am obtinut unele cunostinte de baza pentru utilizarea acesteia, ceea ce sunt sigur ca imi va fi de ajutor pe viitor.  Crearea intregului proiect a fost o experienta interesanta si sper ca sa am ocazia sa mai utilizez aceste tehnologii si in alte proiecte pe viitor.

\clearpage